\KOMAoptions{
    chapterprefix=true,
    numbers=enddot,
}

% Formato de los títulos de capítulos
% Eliminar el punto y reducir el espacio entre el prefijo y el nombre del capítulo 
\renewcommand*{\chapterformat}{%
    \mbox{\chapappifchapterprefix{\nobreakspace}\thechapter
    \IfUsePrefixLine{}{\enskip}}}
\renewcommand{\chapterheadmidvskip}{\vskip 0pt}

% Redefinición de las entradas de los capítulos en la tabla de contenidos
% para que aparezca el prefijo.
\let\originaladdchaptertocentry\addchaptertocentry
\renewcommand*{\addchaptertocentry}[2]{%
  \IfArgIsEmpty{#1}{% Entrada sin número
    \originaladdchaptertocentry{#1}{#2}%
  }{% Entrada con número
    % Eliminar el número y poner el prefijo directamente en el título del capítulo
    \originaladdchaptertocentry{}{\chapapp~#1\autodot\space#2}%
  }%
}

% Formato de los títulos en figuras y tablas
\renewcommand*{\figureformat}{\figurename~\thefigure}
\renewcommand*{\tableformat}{\tablename~\thetable}

% Separador en los títulos de listados de código
\usepackage{caption}
\captionsetup[codelisting]{labelfont=bf, labelsep=default}

% Sangría de los párrafos
\setlength{\parindent}{0.5cm}

% Ajustar símbolos en listas no numeradas
\newfontfamily{\lmromanfont}{Latin Modern Roman}
\setkomafont{itemizelabel}{\lmromanfont}
\renewcommand{\labelitemii}{$\circ$}
\renewcommand{\labelitemiii}{$\diamond$}
\renewcommand{\labelitemiv}{$\triangleright$}